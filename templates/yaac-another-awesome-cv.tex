% !TEX TS-program = luatex
% Awesome Source CV LaTeX Template
%
% This template has been downloaded from:
% https://github.com/darwiin/awesome-neue-latex-cv
%
% Author:
% Christophe Roger (Darwiin)
%
% Template license:
% CC BY-SA 4.0 (https://creativecommons.org/licenses/by-sa/4.0/)

\documentclass[localFont,alternative,10pt,compact]{yaac-another-awesome-cv}
\name{Christophe}{Roger}
\tagline{Architecte Logiciel | Développeur/Concepteur Senior Java/JEE}
\photo{2.5cm}{darwiin}
\socialinfo{
	\linkedin{christopheroger}
	\github{darwiin}\\
	\smartphone{+687 831 831}
	\email{christophe.roger@mail.com}\\
	\address{17 Rue de Bretagne, 98835 Dumbéa, Nouvelle-Calédonie}\\
	\infos{Né le 23 septembre 1982 (35 ans) à Nouméa, Nouvelle-Calédonie}
}

% Uncomment the following line and use a value from 1.5cm to 2.5cm
%\setleftcolumnlength{2.5cm}



\begin{document}

	\makecvheader

	\makecvfooter
		{\textsc{\today}} % \selectlanguage{english}\today
		{\textsc{Christophe Roger - CV}}
		{\thepage}

	\par{
	Développeur et concepteur JEE dès la fin de mes études, j'ai une expérience de développement sur l'ensemble de l'écosystème Java (Android, J2ME sur PDA et Javacard sur chipset NFC). J'occupe aujourd'hui un poste d'architecte logiciel et suis passionné par mon métier et par les nouvelles technologies en général. Je souhaite aujourd'hui poursuivre ma carrière sur des projets de développement innovants en qualité d'architecte logiciel et/ou de développeur/concepteur.
	}
	\sectionTitle{Compétences}{\faTasks}

		\begin{keywords}
			\keywordsentry{Language de programmation}{\textbf{Java} (\textbf{JEE}, \textbf{JSE}, JME, Java Card Platform), Microsoft .Net (\textbf{C\#}), \textbf{Typescript}, Javascript}
			\keywordsentry{Frameworks}{\textbf{Ionic}, Spring Boot, Spring, Angular, ZK}
			\keywordsentry{Bases de données}{IBM DB2, Oracle Database, Microsoft SQL Server, MySQL, PostgreSQL}
			\keywordsentry{Outils de développement}{\textbf{IntelliJ Idea}, \textbf{Visual Studio Code}, Eclipse, Maven, svn, git}
			\keywordsentry{Middleware}{\textbf{JBoss EAP}, \textbf{Apache Tomcat}, Websphere Application Server (\emph{WAS})}
			\keywordsentry{Systèmes d'exploitation}{Mac OS X, Windows Server, Windows 7, Linux Redhat, Linux CentOS}
			\keywordsentry{Autres}{Firebase, architecture SOA, RFID, NFC et code barre 1D/2D}
		\end{keywords}
	\sectionTitle{Expériences Professionelles}{\faSuitcase}
	%\renewcommand{\labelitemi}{$\bullet$}
	\begin{experiences}
	  \experience
	    {Aujourd'hui}   {Architecte logiciel | Développeur/Concepteur Senior JEE}{EPI}{Nouvelle-Calédonie}
	    {Décembre 2015} {
	                      \begin{itemize}
	                        \item Applications hybrides : Conception et développement
	                        \item Développements de micro services REST avec Spring Boot : Conception et développement
	                        \item Product Owner : Projet Simply City
	                        \item Reconstruction de la plateforme d'intégration
	                        \item Migration des projets Java sous Maven
	                        \item Evolutions et corrections des bugs du framework de développement interne  
	                      \end{itemize}
	                    }
	                    {Spring Boot,IntelliJ Idea,Ionic 3,Typescript,Firebase,Apache Tomcat,Eclipse,Maven,Jenkins,Nexus}
	  \emptySeparator
	  \experience
	    {Novembre 2015} {Architecte logiciel | Développeur/Concepteur Senior JEE}{\link{http://www.cafat.nc}{CAFAT}}{Nouvelle-Calédonie}
	    {Avril 2014}    {
	                      \begin{itemize}
	                        \item Support et encadrement technique des équipes de développement                           
	                        \item Suivi, validation et intégration des développements externalisés                        
	                        \item Implémentation, analyse et livraison de correctifs de bugs sur les applicatifs métiers  
	                        \item Evolutions et corrections des bugs du framework de développement interne                
	                        \item Rédaction des dossiers d'architecture en collaboration avec les architectes fonctionnels
	                        \item Veille technologique                                                                    
	                      \end{itemize}
	                    }
	                    {IntelliJ Idea,JBoss EAP,Eclipse,Maven,Jenkins,Nexus}
	  \emptySeparator
	  \experience
	    {Mars 2014}     {Architecte logiciel | Développeur/Concepteur Senior JEE}{Bull SAS}{France}
	    {Avril 2012}    {
	                      \begin{itemize}
	                        \item Reconstruction du dépôt fiduciaire de logiciels de Bull Coriolis : réalisation, coordination et reporting
	                        \item Migration du serveur métier vers Open Cobol : suivi de projet et reporting                
	                        \item Solution documentaire collaborative (wiki) : mise en place et formation                   
	                        \item Evolutions et corrections : analyse, conception et développement                          
	                        \item Mise en place de conventions de code                                                      
	                        \item Mise en place d'un framework de développement d'interface web (jQuery, Bootstrap, taglibs)
	                      \end{itemize}
	                    }
	                    {Tomcat,Spring,Eclipse,Maven,Oracle DB,Hibernate,RichFaces,AngularJS,jQuery,Bootstrap,LESS}
	  \emptySeparator
	  \consultantexperience
	  {Mars 2012}       {Ingénieur Consultant}{Altran Technologies}{France}
	  {Décembre 2007}   {IT Specialist}{IBM, Software Solutions Center of Excellence}
	                    {
	                      \begin{itemize}
	                        \item \textbf{Projet eTACT} pour \href{https://www.edqm.eu/fr/contexte-mission-cd-p-phcmed.html}{EDQM} : Conception et développement JEE.
	                        \item Application \emph{Android} pour tablette : Conception et développement.
	                        \item Projets d'intégration, \emph{Enterprise Service Bus} (ESB) et moteur de processus:
	                        \begin{itemize}
	                          \item Conception et développement JEE
	                          \item Définition et implémentation des processus métiers et médiations
	                        \end{itemize}
	                        \item Solutions RFID : Conception et développement Java (JEE, JSE et JME), Analyse, \emph{POC}, documentation et présentation technique du protocole ONS
	                      \end{itemize}
	                    }
	                    {Rational Software Architect (\emph{RSA}),Eclipse,\emph{WAS} 7,DB2,Hibernate,Ant,RichFaces,Infosphere Traceability Server,Android,Websphere Integration Developer,Websphere Process Server}
	  \emptySeparator         
	  \experience
	  {Novembre 2007}  {Ingénieur d'étude}{IBM}{France}
	  {Février 2007}   {
	                      Projet de prototypage \emph{Campus Nova} pour le Crédit Agricole : Développement d'une solution de paiement NFC sur téléphones portables.  
	                      \begin{itemize}
	                        \item Implémentation d'un porte monnaie électronique                                            
	                        \item Intégration avec une plateforme de paiement en ligne  
	                      \end{itemize}
	                  }
	                  {J2ME,Java Card,DB2,\emph{WAS}}  
	\end{experiences}

	%Section: Languages
	\twocolumnsection
	{\sectionTitle{Langues}{\faLanguage}
	\begin{skills}
		\skill{Français}{5}
		\skill{Anglais}{4}
	\end{skills}}
	{\sectionTitle{Forces}{\faPlus}
	\vspace{1em}
	\begin{itemize}
		\item Passioné
		\item Motivé                    
	    \item Autonome
	\end{itemize}
	}
	\sectionTitle{Formation}{\faGraduationCap}
	\begin{scholarship}
		\scholarshipentry{2007}
						{Master STIC Professionel filière MBDS de l'Université de Nice Sophia Antipolis (Master Informatique spécialité Multimédia, Base de Données et intégration de Systèmes)}
		\scholarshipentry{2005}
						{Licence Sciences et Technologies, Mention Informatique, de l'Université de Nouvelle-Calédonie}
		\scholarshipentry{2004}
						{BTS Informatique de Gestion option administrateurs de réseaux}
		\scholarshipentry{2000}
						{Baccalauréat Scientifique option Mathématiques}
	\end{scholarship}

	%Section: Centres d'intérêt
	\section{\texorpdfstring{\color{Blue}Centres d'intérêts}{Centres d'intérêts}}
	\begin{tabular}{rl}
	    \textsc{Développement mobile:} & iOS, Android, \textbf{Windows Phone}\\
	    \textsc{Développement web:} & HTML5, CSS3 \\ 
	    \textsc{Photographie} & \\
	\end{tabular}
	\sectionTitle{Projets}{\faLaptop}

	\begin{projects}
		\project
		{Simply City}{2017 - 2018}
		{\website{https://www.simplycity.nc}{https://www.simplycity.nc} \website{https://innovation.engie.com/fr/news/actus/territoires/simply-city-lappli-qui-simplifie-la-ville-au-ces-2018-avec-engie/8156}{Présentation CES 2018}}
		{Simply City est une application mobile, gratuite et participative destinée à tous les habitants, visiteurs et touristes qui séjournent dans une ville. L’application permet de connaître toutes les informations et services utiles en temps réel.}
		{Ionic 3,Typescript,Javascript,Visual Studio Code}
				
		\project
		{YAAC Another Awesome CV}{2013 - 2018}
		{\github{darwiin/yaac-another-awesome-cv} \website{https://www.overleaf.com/latex/templates/awesome-source-cv/wrdjtkkytqcw}{Template sur Overleaf}}
		{Template \LaTeX pour Curiculum Vitæ utilisant les icônes \href{https://fontawesome.com}{Font Awesome} et la police de caractère \href{https://fonts.google.com/specimen/Source+Sans+Pro}{Adobe Source Sans Pro}. YAAC Another Awesome CV a d'abord été créé comme un template simple pour CV à vocation technologique.}
		{\LaTeX,Sublime Text}

	\end{projects}

	%\sectionTitle{Projets}{\faLaptop}
	%\twocolumnsection{
	%	\begin{projects}
	%		\project
	%			{YAAC Another Awesome CV}{2013 - 2018}
	%			{\github{darwiin/yaac-another-awesome-cv} }
	%			{Template \LaTeX pour la réalisation de Curiculum Vitæ qui utilise \href{https://fontawesome.com}{Font Awesome} et la police de caractère Adobe Source.}
	%		{\LaTeX,Sublime Text}
	%	\end{projects}
	%}
	%{\begin{projects}
	%	\project
	%	{Simply City}{2017 - 2018}
	%	{\github{darwiin} \website{https://www.simplycity.nc}{https://www.simplycity.nc}}
	%	{Igitur nam locis plane homines quidem et locis dicit quot quidem quod fallare si sed satisfacit intellegam et falli dicit mihi igitur possumus locis admodum eloquentiam vult et non admodum complectitur quod intellegam et intellegam complectitur et tamen philosopho quidem si vult igitur locis falli ego non philosophi habeat Torquate et non inquam pluribus asperner verbis dicit igitur sententiae locis non si sententiae oratio et non quot aeque habeat non homines habeat si inquam quod ego istius istius quidem ego pluribus aeque oratio non quidem pluribus verbis si igitur tamen nam istius vult non tot istius et ego non non.}
	%	{DB2,Eclipse,Infosphere Traceability Server}
	%	\end{projects}
	%}
	\sectionTitle{Références}{\faQuoteLeft}

	\begin{referees}
		\referee
			{Jon Snow}
			{Lord Commander}
			{Night's Watch}
			{john.snow@nightwatch.org}
			{+687 987 654}
		\referee
			{Géry Loutre}
			{Architecte logiciel}
			{Cafat}
			{ref1@cafat.nc}
			{+687 987 654}
	\end{referees}


\end{document}